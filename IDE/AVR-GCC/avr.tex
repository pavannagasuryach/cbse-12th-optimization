\documentclass[10pt,a4paper]{article}
\usepackage[a4paper,outer=2cm,inner=2cm,top=2cm,bottom=2cm]{geometry}
\usepackage{tabularx}
\usepackage{amsmath}
\begin{document}
\centerline{\textbf{IDE ASSIGNMENT}}
\centerline{AVR-GCC}
\centerline{Pavan NagaSurya Cheemalamarry}
\centerline{pavannagasuryach18.555@gmail.com}
\centerline{IITH - Future Wireless Communication (FWC22137)}
\tableofcontents
\section{problem}
(GATE2020-QP-EE)\\
Q.54 An 8085 microprocessor accesses two memory locations (2001H) and (2002H), that contain 8-bit numbers 98H and
 B1H, respectively.\\
 Thhe following program is executed:\\
 LXI H, 2001H\\
 MVI A, 21H\\
 INX H\\
 ADD M\\
 INX H\\
 MOV M, A\\
 HLT\\
 At the end of this program, the memory location 2003H contains the number in decimal (base10) from\\
\section{components}]
\begin{tabularx}{0.8\textwidth}{
		| >{\centering\arraybackslash}X
		| >{\centering\arraybackslash}X
		| >{\centering\arraybackslash}X |}
	\hline
	Components & Value & Quantity\\
	\hline
	Breadboard & & 1\\
	\hline
	Arduino & uno & 1\\
	\hline
	Jumper Wires & & 4\\
	\hline
\end{tabularx}
\subsection{Arduino}
The Arduino Uno has some ground pins. analog input pins A0-A3 and digital pins D1-D13 that can be used for both 
input as well as output. It also has two power pins that can generate 3.3V and 5V . In the following exercise,
 We use digital pins, GND and 5V
 \section{Implementation}
 \subsection{Equation}
	LXI, 2001H;H = 20 H, L =01 H \\
	MVI A,21H;A = 21 H \\
	INX H;HL + 1 $ \rightarrow $ H =20 H, L = 02 H $ \rightarrow $ HL = 2002 H \\
	ADD M; [A] + Refeerence data of HL pair = 21 H + B1 H = D2H $ \rightarrow $ [A] \\
	INX H;  [HL] +1 $ \rightarrow $ 002 H + 1 H $ \rightarrow $ 2002 H \\
	MOV M,A;[A] to  memory, reference of HL pair, 2003 H [D2]  [D2] =A \\
	HLT; Stop \\
	Converting in decimal \\
	210 \\
\section{Hardware}
\begin{enumerate}
	\item  Connect one end of jumper wire to the ground pin on the Arduino and other end to the breadboard's 
		ground rail.
	\item  Connect theone terminal of jumper wire to the input pin of Arduino and other end to the positive 
		rail on breadboard.
	\item Connect one end of another jumper wire to the input pin of Arduino and other end to the positive 
		rail.
	\item Enable the power supply to breaboard from Arduino by connecting one end of jumper wire to the 
		power pin of arduino and other end to the positive rail on breadboard.
\end{enumerate}
\section{Conclusion}
  Hence, we have implemented the above problem using the code below : \\
  \framebox{https://github.com/pavannagasuryach/cbse-12th-optimization/tree/main/IDE/AVR-GCC}
\end{document}
