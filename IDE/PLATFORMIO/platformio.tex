\documentclass[10pt,a4paper]{article}
\usepackage[a4paper,outer=2cm,inner=2cm,top=2cm,bottom=2cm]{geometry}
\usepackage{multicol}
\usepackage{listings}
\usepackage{circuitikz}
\usepackage{amsmath}
\usepackage{setspace}
\usepackage{tabularx}
\begin{document}
 \centerline{\textbf{\large IDE ASSIGNMENT}}
 \centerline{\textbf{\large PLATFORMIO}}
 \centerline{PAVAN NAGASURYA CHEEMALAMARRY}
 \centerline{pavannagasuryach18.555@gmail.com}
\centerline{IITH - FUTURE WIRELESS COMMUNICATIONS( FWC22137 )}
\setstretch{1.5}
\tableofcontents
	\section{Question}
		In the circuit shown below , X and Y are digital inputs, and Z is a digital output. The quivalent circuit is a \\
		
	     	\begin{center}
		\centering
		\begin{circuitikz}[scale=1]
    % 1st not
        \draw (0,0) node[not port,scale=0.8 ] (not) {};
        \draw (3,-0.28) node[and port] (and) {};
        
        \draw (not.in 1) -- ++(-1,0) node[left] {$X$};
        \draw (not.out) -- ++(0,0) coordinate (and.in 1);
        \draw (not.in 1) -- ++(0,-1.7) node[below] {$ $};
    % 1st And
        \draw (and.in 1) -- ++(-1.2,0) node[left] {$ $};
        \draw (and.in 2) -- ++(-3.3,0) node[left] {$Y$};
        \draw (and.out) -- ++(0,0) node[right] {$ $};

        \draw (and.in 2) -- ++(-3,0) coordinate (point);
        \draw (point) -- ++(0,-1.7) -- ++(1,0) node[below] {$ $};
        \draw (and.out) -- ++(0,-0.47) node[below] {$ $};
    %2nd not
        \draw (0,-2.28) node[not port ,scale=0.8] (not) {};
        \draw (not.in 1) -- ++(-0.8,0) node[left] {$ $};
        \draw (not.out) -- ++(0,0) coordinate (and.in 2);
   %2nd ANd
        \draw (3,-2) node[and port] (and) {};
        \draw (and.in 1) -- ++(-2.2,0) node[left] {$ $} ;
        \draw (and.in 2) -- ++(-1.2,0) node[left] {$ $} ;
        \draw (and.out) -- ++(0,0) node[right] {$ $};
          \draw (and.out) -- ++(0,0.7) node[above] {$ $};
    %or
        \draw (6,-1) node[or port] (or) {};
        \draw (or.in 1) -- ++(-1.48,0) node[left] {$ $} ;
        \draw (or.in 2) -- ++(-1.48,0) node[left] {$ $} ;
        \draw (or.out) -- ++(1,0) node[right] {$Z$};
  
     \end{circuitikz}
		\end{center} 
		\begin{multicols}{4}
			\begin{enumerate}
				\item NAND gate
				\item NOR gate
				\item XOR gate
				\item XNOR gate
			\end{enumerate}
		\end{multicols}

	\section{components}
		\begin{table}[htbp]
		\centering
			\begin{tabularx}{1\textwidth}
			{
				| >{\centering\arraybackslash}X
				| >{\centering\arraybackslash}X
				| >{\centering\arraybackslash}X |}
			\hline
			{\bf Components} & {\bf Value} & {\bf Quantity} \\
			\hline
			Arduino & Uno & 1\\
			\hline
			BreadBoard & &  1 \\
			\hline
			Jumper Wires & & 4 \\
			\hline
		\end{tabularx}
			\caption{Components}
			\label{table=Components}
		\end{table}

	\section{Implementation}
				\subsection{Boolean Expression}
 				By solving above expression we get :
				\begin{align}
					z=& x' . y +x.y' \nonumber \\
					z=& x'y+xy' \nonumber  \end{align} 
			\subsection{Truth Table}

                        \begin{table}[htbp] 
				\centering  
				\begin{tabularx}{0.5\textwidth}
                        {  | >{\centering\arraybackslash}X
                           | >{\centering\arraybackslash}X
                           | >{\centering\arraybackslash}X |}
                         \hline
                         {\bf A} & {\bf B} & {\bf OUT} \\
                       \hline 
			0 & 0 & 0\\   
			\hline  
			0 & 1 & 1 \\
                       \hline                                                                                      
	         	1 & 0 & 1 \\  
		        \hline   
			1 & 1 & 0 \\
			\hline
			\end{tabularx}
                        \caption{Truth Table}
				\label{table=truth}
			\end{table}
     \section{Hardware}
	     \begin{enumerate}
		     \item Make the connenctions between the arduino and bread board as shown in Table3.
			     \begin{table}[h]
				     \centering
				     \begin{tabularx}{0.5\textwidth}
					     {
						     | >{\centering\arraybackslash}X
						     | >{\centering\arraybackslash}X
						     | >{\centering\arraybackslash}X |}
						     \hline
						     {\bf Arduino} & 5.0v & GND \\
						     \hline
						     {\bf Bread Board} & +ve & -ve  \\
						     \hline
				     \end{tabularx}
			     \end{table}
			     \item Connect one end of a jumper wire to the GND(ground) pin on the Arduino Uno board
				     and  other end to the   breadboard’s ground rail(-).
                             \item Connect one terminal of jumper wire (Input A) to the input pins on the Arduino
				     (e.g., pin2) and other terminal to
				     the positive rail(+) on the breadboard.
			     \item Connect one end of another jumper wire (Input B) to the input pin of Arduino
				     (e.g., pin3) and other end to the 
				     positive rail(+) on the breadboard.
			     \item Enable the power supply to breadboard from arduino by connecting one end of 
			             jumper wire to the power pin of Arduino(5V) and other end to the positive 
				     rail(+) on the breadboard.
	     \end{enumerate}
     \section{Conclusion}
	     Hence , we have implemented the XOR gate by the code . given below : \\
	     \framebox{https://github.com/pavannagasuryach/cbse-12th-optimization/tree/main/IDE/PLATFORMIO}

\end{document}
